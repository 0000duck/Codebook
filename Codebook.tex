%!TEX TS-program = xelatex
%!TEX encoding = UTF-8 Unicode

% 一定要用 xelatex 進行編譯,字體目前是使用Mac內建的中文字體

%%%%%%%%%%%%%%%%頁面設定%%%%%%%%%%%%%%%%%%%

\documentclass[10pt,twocolumn,oneside]{article} %垂直兩欄、單面列印
\setlength{\columnsep}{14pt} %兩欄模式的間距
\setlength{\columnseprule}{0pt} %兩欄模式間格線粗細

\linespread{1} % 行距

\usepackage{enumitem} % Disable spacing in enumeration list
\setlist{nolistsep}

\usepackage{geometry} %定義邊界
\geometry{a4paper, total={170mm,257mm}, left=20mm, top=20mm}

%%%%%%%%%%%%%%%%%%%%%%%%%%%%%%%%%%%%%%%

\usepackage{pdfpages} %導入pdf文檔, \includepdf[pages={3,4}]{problem_statement.pdf}

\usepackage[colorlinks=true, urlcolor=cyan]{hyperref} %使用超連結, \href{http://codeforces.com/blog/entry/17974}{Codeforces}

\usepackage{fancyhdr}	%設定頁首頁尾

\usepackage{verbatim}  % 大量註解用 , \begin{comment}  % 註解開始,  \end{comment}  % 註解結尾

%%%%%%%%%%%%%%%%中文設定%%%%%%%%%%%%%%%%%%%

\usepackage{fontspec} %加這個就可以設定字體 
\usepackage{xeCJK} %讓中英文字體分開設置
%\setmainfont{Arial}
\setCJKmainfont{STKaiti} %設定中文為系統上的字型,而英文不去更動,使用原TeX字型
\setmonofont{Courier} %等寬字體
\XeTeXlinebreaklocale "zh" %這兩行一定要加,中文才能自動換行
\XeTeXlinebreakskip = 0pt plus 1pt 

%%%%%%%%%%%%%%%程式碼設定%%%%%%%%%%%%%%%%%%%

\usepackage{listings}
\usepackage{xcolor}

\definecolor{mygreen}{rgb}{0,0.6,0}
\definecolor{mygray}{rgb}{0.5,0.5,0.5}
\definecolor{mymauve}{rgb}{0.58,0,0.82}

\lstset{ %
  backgroundcolor=\color{black!5}, % set backgroundcolor,  choose the background color; you must add \usepackage{color} or \usepackage{xcolor}
  basicstyle=\footnotesize,   % the size of the fonts that are used for the code
  breakatwhitespace=false,   % sets if automatic breaks should only happen at whitespace
  breaklines=true,  % sets automatic line breaking
  captionpos=b,  % sets the caption-position to bottom
  frame=single,	  % adds a frame around the code
  escapeinside={\%*}{*)},  % if you want to add LaTeX within your code
  keepspaces=true,  % keeps spaces in text, useful for keeping indentation of code (possibly needs columns=flexible)
  columns=flexible,
  commentstyle=\color{mygreen}, % comment style
  keywordstyle=\color{blue},  % keyword style
  %otherkeywords={*,...}, % if you want to add more keywords to the set  
  numbers=left, % where to put the line-numbers; possible values are (none, left, right)
  numbersep=7pt, % how far the line-numbers are from the code
  numberstyle=\tiny\color{mygray}, % the style that is used for the line-numbers
  stepnumber=1,  % the step between two line-numbers. If it's 1, each line will be numbered
  rulecolor=\color{black}, % if not set, the frame-color may be changed on line-breaks within not-black text (e.g. comments (green here))
  showspaces=false, % show spaces everywhere adding particular underscores; it overrides 'showstringspaces'
  showstringspaces=false, % underline spaces within strings only
  showtabs=false, % show tabs within strings adding particular underscores
  stringstyle=\color{mymauve},     % string literal style
  tabsize=4, % sets default tabsize to 4 spaces
  basicstyle=\footnotesize\ttfamily, %等寬字體 字體大小
  title=\lstname % show the filename of files included with \lstinputlisting; also try caption instead of title
}

\lstdefinestyle{customc}{
  belowcaptionskip=1\baselineskip,
  breaklines=true,
  frame=L,
  xleftmargin=\parindent,
  language=C++,
  showstringspaces=false,
  keywordstyle=\bfseries\color{green!40!black},
  commentstyle=\itshape\color{purple!40!black},
  identifierstyle=\color{blue},
  stringstyle=\color{orange},
}

%%%%%%%%%%%%%%%%%%%%%%%%%%%%%%%%%%%%%%%

\begin{document} 

%%%%%%%%%%%%%%%%%頁首%%%%%%%%%%%%%%%%%%%%

% generate header and footer
\pagestyle{fancy}
\fancyhead[L]{National Chung Cheng University -- Earthrise}
\fancyhead[R]{\thepage}

%\fancyfoot[L]{\includegraphics[width=20pt]{pic.jpg}} %team profile picture
\fancyfoot[C]{\today}
\fancyfoot[R]{\thepage}

% generate table of content on in a new page
\scriptsize
\tableofcontents
\newpage

%%%%%%%%%%%%%%%%%正文%%%%%%%%%%%%%%%%%%%%

\section{Todo}
\begin{enumerate}
\item Add code and complexity
\item Add brief explanations
\end{enumerate}

%%%%%%%%%%%%%%%%機器設定%%%%%%%%%%%%%%%%%%

\section{Contest Setup}

\subsection{vimrc}
\lstinputlisting{contest_setup/vimrc}

\subsection{bashrc}
\lstinputlisting[language=bash]{contest_setup/bashrc}

\newpage

\subsection{C++ template}
\lstinputlisting[style=customc]{contest_setup/main.cpp}

\subsection{Java template}
\lstinputlisting[style=customc]{contest_setup/Main.java}

\newpage

%%%%%%%%%%%%%%%%%忠告%%%%%%%%%%%%%%%%%%%%

\section{Reminder}

\begin{enumerate}
	\item Read the problem statements carefully. Input and output specifications are crucial!
	\item Estimate the \textbf{time complexity} and \textbf{memory complexity} carefully.
	\item Time penalty is 20 minutes per WA, \textbf{don't rush}!
	\item Sample test cases must all be tested and passed before every submission!
	\item Test the corner cases, such as 0, 1, -1. Test all edge cases of the input specification.
\end{enumerate}

%%%%%%%%%%%%%%%%實用代碼%%%%%%%%%%%%%%%%%%

\section{Useful code}

\subsection{Fast Exponentiation}

\subsection{GCD}

小心負數!

\subsection{Extended Euclidean Algorithm}

\subsection{STL quick reference}

\subsubsection{Map / Set}

\subsubsection{String}

%%%%%%%%%%%%%%%%%搜尋%%%%%%%%%%%%%%%%%%%%

\section{Search}

\subsection{Binary Search}

\subsubsection{Find key}

\subsubsection{Upper / lower Bound}

\subsection{折半完全列舉}

\subsection{Two-pointer 爬行法}

%%%%%%%%%%%%%%%%資料結構%%%%%%%%%%%%%%%%%%

\section{Basic data structure}

\subsection{1D BIT}

%\lstinputlisting[style=customc]{old/"UVA 10810 BIT.cpp"}

\subsection{2D BIT}

\subsection{Union Find}

%\lstinputlisting[language=c++]{A/main.cpp}

\subsection{Segment Tree}

Hehe

%%%%%%%%%%%%%%%%%%DP%%%%%%%%%%%%%%%%%%%%

\section{Dynamic Programming}

%%%%%%%%%%%%%%%%%%樹%%%%%%%%%%%%%%%%%%%%

\section{Tree}

\subsection{LCA}

%%%%%%%%%%%%%%%%%圖論%%%%%%%%%%%%%%%%%%%

\section{Graph}

\subsection{Articulation point / edge}

\subsection{BCC vertex}

\subsection{BCC edge}

\subsection{SCC}

\subsection{Shortest Path}

\subsubsection{Dijkatra}

\subsubsection{SPFA}

\subsubsection{Bellman-Ford}

\subsection{Flow}

\subsubsection{Max Flow (Dinic)}

\subsubsection{Min-Cut}

\subsubsection{Min Cost Max Flow}

\subsubsection{Maximum Bipartite Graph}

%%%%%%%%%%%%%%%%%字串%%%%%%%%%%%%%%%%%%%

\section{String}

\subsection{KMP}

\subsection{Z Algorithm}

\subsection{Trie}

\subsection{Suffix Array}

%%%%%%%%%%%%%%%%%幾何%%%%%%%%%%%%%%%%%%%

\section{Geometry}

\subsection{Template}

\subsubsection{Point / Line}

\subsubsection{Intersection}

\subsection{Half-plane intersection}

\subsection{Convex Hull}

\end{document}
